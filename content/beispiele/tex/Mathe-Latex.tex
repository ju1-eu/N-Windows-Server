\section{\textbf{Text Unterstreichen}}
    \underline{wichtiger Text} und \emph{kursiver Text} und \textbf{fetter Text}

 
\section{\textsc{Kapitaelchen}}  
    Text in Grossbuchstaben setzen durch \LaTeX
    \begin{itemize} % \item[] avoids bullet
        \item[] Einen längeren Satz\\ einrücken.
    \end{itemize}

\section{Vor- und Nachteile}
    \begin{tabular}[h]{ll}
        {\textbf{Vorteile}}   &  {\textbf{Nachteile}} \\
        Argument 1            &  Argument 2 \\
        Argument A            &  Argument B \\
    \end{tabular} 

\section{Summe}
    \begin{tabular}[h]{clrr}
        & Betrag &               &  $1.000,00$ \\
    $-$ & Skonto & $2\%$         &     $20,00$ \\
        \hline
    $=$ & $\sum$ &               &    $980,00$  
    \end{tabular} 
    
\section{Division Zinsen}
    $
        \frac{\text{ Betrag } \cdot \text{ Prozentsatz } \cdot \text{ Zeit } }{100 \cdot \text{ Zeitgröße }} 
        = \frac{12.597,90 \cdot 12 \cdot 20 \text{ T } }{100 \cdot 360} 
        = 83,99 \text{ \euro }
    $

    \begin{align*}
        \frac{\text{ Betrag } \cdot \text{ Prozentsatz } \cdot \text{ Zeit } }{100 \cdot \text{ Zeitgröße }} 
        = \frac{12.597,90 \cdot 12 \cdot 20 \text{ T } }{100 \cdot 360} 
        = 83,99 \text{ \euro }
    \end{align*} 

\section{Tabelle 2}
    \begin{table}[!ht]% hier: !ht 
        \begin{tabular}{@{}lcr@{}}
        \toprule 
        \textbf{Großbuchstaben} & \textbf{Kleinbuchstaben} & \textbf{Name}\\
        \midrule
        21 \euro & 22000 & 230.000 \\
        $31$ \euro & $32000$ & $330.000$ \\
        \bottomrule
        \end{tabular}
    \end{table}

\section{Checkliste}
    Meine Liste
    \begin{itemize} \itemsep -2pt  % reduce space between items
        \item[$\Box$]   Punkt 1
        \item[$\Box$]   Aufgabe 2 
    \end{itemize}

    \begin{itemize}[label=\checkmark] \itemsep -2pt
        \item Check 1
        \item Check 2   
    \end{itemize}

\section{\textcolor{rot5}{Nenne 4x Lerzielstufen (Taxonomien)}}
    \begin{enumerate}[label={\protect\ding{\value*}},start=192]
        \item Reproduktion
        \item Reorganisieren
        \item Transfer
        \item Kreativ
    \end{enumerate}

\section{Wurzel berechnen}
    \begin{multicols}{3}
        \begin{enumerate}[label=(\alph*)]
            \item $\sqrt{169}$
            \item $\sqrt{0,36}$
            \item $\frac{\sqrt{45}}{\sqrt{80}}$
        \end{enumerate}
    \end{multicols}

\section{Aufgabe Logik}
    Formalisieren Sie die folgenden Aussagen und verneinen Sie sie anschließend (ohne das Wort nicht davor zu setzen) und übersetzen Sie wieder in Umgangssprache:

    \begin{enumerate}[label=(\alph*)]
        \item Volksmund: >>Bei Nacht sind alle Katzen grau.<<
        \item Plakatwerbung: >>Wenn einer hochguckt, dann gucken alle.<<
        \item Gorbatschov: >>Wer zu spät kommt, den bestraft das Leben.<<
    \end{enumerate}

\section{Quotientenregel}
    \begin{itemize} % \item[] ohne bullet
        \item[] $\left(\frac{u}{v}\right)^{\prime} = \frac{u^{\prime} \cdot v-u \cdot v^{\prime}}{v^{2}}$
        
        \item[] $\frac{\text{ Ableiten } \cdot \text{ Stehen lassen } - \text{ Stehen lassen } \cdot \text{ Ableiten }}{\text{ Nenner}^2}$
    \end{itemize}

    
\section{Logarithmus}
    \begin{align}
        ln(a \cdot b)   &= ln(a) + ln(b) \\
        ln(\frac{a}{b}) &= ln(a) - ln(b) \\
        ln(a^b)         &= b \cdot ln(a)
    \end{align}

    \newpage
\section{\LaTeX - Assistent}
    Formeleditor~\footnote{\url{https://www.matheretter.de/rechner/latex}}
    \begin{align}
        \text{ Matrix }       &= \begin{pmatrix} a & b \\ c & d \end{pmatrix} \\
        \text{ Vektor }       &= \begin{pmatrix} x\\y \end{pmatrix} \\
        \text{ Vektorbuchstabe } &= \vec{x} \\
        \text{ Wurzel }       &= \sqrt[n]{a} \text{ Potenz } a^{b} \\
        \text{ Bruch }        &= \frac{a}{b} \\
        \text{ Log }          &= \log_{b}{a} \\
        \text{ Summe }        &= \sum \limits_{n=0}^{\infty} \\
        \text{ Index }        &= x_{1,2} \\
        \text{ Klammern }     &= \left\{x, y\right\} \\
        \text{ Alphabet kl. } &= +\\%α β γ δ ε ζ η θ ι κ λ μ ν ξ ο π ρ σ τ υ φ χ ψ ω \\
        \text{ Alphabet gr. } &= +\\%Α Β Γ Δ Ε Ζ Η Θ Ι Κ Λ Μ Ν Ξ Ο Π Ρ Σ Τ Υ Φ Χ Ψ Ω \\
        \text{ Element }      &= \in \notin \sum \quad \prod \quad () \quad \to \quad \infty\\
        \text{ Mengen }       &= \mathbb{N} \mathbb{Z} \mathbb{Q} \mathbb{R} \mathbb{I} \mathbb{C} \mathbb{L} \\
        \text{ Relation }     &= < > \geq \leq \neq \subset \subseteq \approx \in \supset \supseteq \notin \\
        \text{ Pfeile }       &= +\\%\rightarrow \leftarrow \Longleftrightarrow \Longrightarrow \Longleftarrow \\
    \end{align}

\section{Formeln in einer Zeile}
    $
        u = \bar{u} + \epsilon \cdot u_1 \quad
        v = \bar{v} + \epsilon \cdot v_1 \quad
        w = \bar{w} + \epsilon \cdot w_1 \quad
    $

    $
        \left( \begin{array}{rrr}
            1 & 0 & 0 \\                                              
            0 & 1 & 0 \\
            0 & 0 & 1 \\                                              
        \end{array}\right)
    $

\section{Sonderzeichen}
    \textbackslash \{...\} \$ \& \# \textdegree \^{} \_ \textasciitilde \%

\section{Währungszeichen}
    \euro 100 \textdollar 100 \textsterling 100 $1.000,00 \text{ \euro }$ 1.000,00 \euro

\section{Leerzeichen}
\begin{itemize} % \item[] avoids bullet
    \item[] [a\,b] ($0.16667em$)
    \item[] [a\:b] ($0.2222em$)
    \item[] [a\enspace b] ($0.5em$)
    \item[] [a\quad b] ($1em$)
    \item[] [a\hspace{5em} b] (5em)
\end{itemize}

\newpage
\section{Griechisches Alphabet}
    \begin{table}[!ht]% hier: !ht 
        \begin{tabular}{@{}ccl@{}}
        \toprule 
        \textbf{Großbuchstaben} & \textbf{Kleinbuchstaben} & \textbf{Name}\\
        \midrule
        A & $\alpha$ & Alpha\\
        B & $\beta$ & Beta\\
        $\Gamma$ & $\gamma$ & Gamma \\
        $\Delta$ & $\delta$ & Delta\\
        E & $\epsilon$, $\varepsilon$ & Epsilon\\
        Z & $\zeta$ & Zeta\\
        H & $\eta$ & Eta\\
        $\Theta$ & $\theta$, $\vartheta$ & Theta\\
        I & $\iota$ & Iota\\
        K & $\kappa$, $\varkappa$ & Kappa\\
        $\Lambda$ & $\lambda$ & Lambda\\
        M & $\mu$ & My\\
        N & $\nu$ & Ny\\
        $\Xi$ & $\xi$ & Xi\\
        O & o & Omikron\\
        $\Pi$ & $\pi$, $\varpi$ & Pi\\
        P & $\rho$, $\varrho$ & Rho\\
        $\Sigma$ & $\sigma$, $\varsigma$ & Sigma\\
        T & $\tau$ & Tau\\
        Y & $\upsilon$ & Ypsilon\\
        $\Phi$ & $\phi$, $\varphi$ & Phi\\
        X & $\chi$ & Chi\\
        $\Psi$ & $\psi$ & Psi\\
        $\Omega$ & $\omega$ & Omega\\
        \bottomrule
        \end{tabular}
    \end{table}

\section{Tabelle 3}
    Tabellengenerator~\footnote{\url{https://www.tablesgenerator.com/}} 
    und Rechner~\footnote{\url{https://www.matheretter.de/rechner/latex}}

    \begin{multicols}{2}
        \begin{tabular}[h]{ll|l}
            &  A     & B     \\ 
        \hline
        1)* &  $a_1$ & $b_1$ \\
        2)  &  $a_2$ & $b_2$ \\
        3)  &  $a_3$ & $b_3$ 
        \end{tabular}
        
        \columnbreak% Spalte
        *Beachte: $\sqrt[n]{x} = x^\frac{1}{n}$       
    \end{multicols}

\section{Tabelle und Grafik}
    \begin{multicols}{2}
        \begin{tabular}[h]{l|c|r}
            Spalte 1 & Spalte 2 & Spalte 3 \\
            \hline
            heise & tipps & tricks \\
        \end{tabular}    

        \columnbreak% Spalte

        \includegraphics[width=2.0cm]{images/logo.eps}% Logo   
    \end{multicols}  

\newpage
\section{Farben}
    \begin{testcolors}[rgb,cmyk,HTML,gray]
        \testcolor{black}
        \testcolor{white}
        \testcolor{darkgray}
        \testcolor{gray}
        \testcolor{lightgray}
        \testcolor{red}
        \testcolor{green}
        \testcolor{blue}
        \testcolor{cyan}
        \testcolor{magenta}
        \testcolor{yellow}
        \testcolor{brown}
        \testcolor{lime}
        \testcolor{olive}
        \testcolor{orange}
        \testcolor{pink}
        \testcolor{purple}
        \testcolor{teal}
        \testcolor{violet}
        \testcolor{rot5}
        \testcolor{blau5}  
        \testcolor{grau2}    
        \testcolor{orange}                       
    \end{testcolors}
    
\section{Farbenfolgen}
    \definecolorseries{test}{rgb}{step}[rgb]{.95,.85,.55}{.17,.47,.37}
    \definecolorseries{test}{hsb}{step}[hsb]{.575,1,1}{.11,-.05,0}
    \definecolorseries{test}{rgb}{grad}[rgb]{.95,.85,.55}{3,11,17}
    \definecolorseries{test}{hsb}{grad}[hsb]{.575,1,1}{.987,-.234,0}
    \definecolorseries{test}{rgb}{last}[rgb]{.95,.85,.55}[rgb]{.05,.15,.55}
    \definecolorseries{test}{hsb}{last}[hsb]{.575,1,1}[hsb]{-.425,.15,1}
    \definecolorseries{test}{rgb}{last}{red!50}{blue}
    \definecolorseries{test}{hsb}{last}{yellow!50}{black}
    \definecolorseries{test}{cmy}{last}{orange!50}{green}

    \resetcolorseries[12]{test}
    \rowcolors[\hline]{1}{test!!+}{test!!+}
    \setlength{\tabcolsep}{5mm} % Abstände zwischen den Spalten
    \begin{tabular}[h]{c||c||c||c||c||c||c||c||c||c}
        $S_1$ & $S_2$ & $G_1$ & $G_2$ & $L_1$ & $L_2$ & $L_3$ & $L_4$ & $L_5$ \\
        \hline \hline
        \number\rownum & \number\rownum & \number\rownum & \number\rownum & \number\rownum & \number\rownum & \number\rownum & \number\rownum & \number\rownum \\
        \number\rownum & \number\rownum & \number\rownum & \number\rownum & \number\rownum & \number\rownum & \number\rownum & \number\rownum & \number\rownum \\
        \number\rownum & \number\rownum & \number\rownum & \number\rownum & \number\rownum & \number\rownum & \number\rownum & \number\rownum & \number\rownum \\
        \number\rownum & \number\rownum & \number\rownum & \number\rownum & \number\rownum & \number\rownum & \number\rownum & \number\rownum & \number\rownum \\
        \number\rownum & \number\rownum & \number\rownum & \number\rownum & \number\rownum & \number\rownum & \number\rownum & \number\rownum & \number\rownum \\
        \number\rownum & \number\rownum & \number\rownum & \number\rownum & \number\rownum & \number\rownum & \number\rownum & \number\rownum & \number\rownum \\
        \number\rownum & \number\rownum & \number\rownum & \number\rownum & \number\rownum & \number\rownum & \number\rownum & \number\rownum & \number\rownum \\
        \number\rownum & \number\rownum & \number\rownum & \number\rownum & \number\rownum & \number\rownum & \number\rownum & \number\rownum & \number\rownum \\
        \number\rownum & \number\rownum & \number\rownum & \number\rownum & \number\rownum & \number\rownum & \number\rownum & \number\rownum & \number\rownum \\
        \number\rownum & \number\rownum & \number\rownum & \number\rownum & \number\rownum & \number\rownum & \number\rownum & \number\rownum & \number\rownum \\
        \number\rownum & \number\rownum & \number\rownum & \number\rownum & \number\rownum & \number\rownum & \number\rownum & \number\rownum & \number\rownum \\
        \number\rownum & \number\rownum & \number\rownum & \number\rownum & \number\rownum & \number\rownum & \number\rownum & \number\rownum & \number\rownum \\
        \number\rownum & \number\rownum & \number\rownum & \number\rownum & \number\rownum & \number\rownum & \number\rownum & \number\rownum & \number\rownum \\
        \number\rownum & \number\rownum & \number\rownum & \number\rownum & \number\rownum & \number\rownum & \number\rownum & \number\rownum & \number\rownum \\
        \number\rownum & \number\rownum & \number\rownum & \number\rownum & \number\rownum & \number\rownum & \number\rownum & \number\rownum & \number\rownum \\ 
    \end{tabular}
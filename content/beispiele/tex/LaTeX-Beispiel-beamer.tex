% ju 23-Jul-21
\section*{Einleitung}

\emph{Sonderzeichen}  wie <<\& oder \%>> müssen mit einem Backslash \verb|\& oder \%| maskiert werden, 
damit sie von LaTeX nicht als Befehle missverstanden werden.


\emph{Website} \footnote{\url{https://golatex.de/wiki/Hauptseite}} \verb|\footnote{\url{https://golatex.de/wiki/Hauptseite}}| 

\emph{PDF Datei einbinden} \verb|\includepdf[landscape=false]{images/logo.eps}| 
\includepdf[landscape=false]{images/logo.eps} 

\clearpage
\subsection*{Stand der Forschung}

Während die traditionelle Latexproduktion bereits hinreichend erforscht ist (\autoref{fig:latex}) \\
\verb|(\autoref{fig:latex})|, bleibt das wissenschaftliche Verständnis elektronischer Verarbeitungsprozesse dieses 
vielseitigen Materials weiterhin lückenhaft. 


\begin{figure}[!ht]% hier: !ht
	\centering
	\includegraphics[width=0.25\textwidth]{images/logo.eps}
	\caption{Traditionelle Latexproduktion}\label{fig:latex}%
\end{figure}

\lstset{language=TeX}% C, TeX, Bash, Python 
\begin{lstlisting}[
	%caption={}, label={code:}%% anpassen
]
% Optionen
scale = Wert, Vergrösserungsfaktor
width/height = Wert für die Einstellung der Breite/Höhe
angle = Wert, Winkel (in Grad)
b = bottom - Seitenende 
t = top - Seitenanfang
h = here
p = page - komplette Seite  
% Abbildung
\begin{figure}[!ht]% hier: !ht
	\centering
	\includegraphics[width=0.25\textwidth]{images/logo.eps}
	\caption{Traditionelle Latexproduktion}\label{fig:latex}%
\end{figure}
\end{lstlisting}


\clearpage
\section*{Methodik}

Unter Zuhilfenahme der Formeln \autoref{eq:ekin} \verb|\autoref{eq:ekin}| und \autoref{eq:impuls} \verb|\autoref{eq:impuls}| werden wir 
diese Forschungslücke schließen.  
$E_{kin}$ \verb|$E_{kin}$| ist die kinetische Energie, $m$ \verb|$m$| die Masse und $\vec{v}$ \verb|$\vec{v}$| die Geschwindigkeit.

Wurzel $\sqrt{2}$ \verb|$\sqrt{2}$|

Bruch $\frac{\text{Zähler}}{\text{Nenner}}$ \verb|$\frac{\text{Zähler}}{\text{Nenner}}$|

\begin{equation}
	\label{eq:ekin}% 
	\sum E_{kin} = \sum E'_{kin}
\end{equation}

\begin{equation}
	\label{eq:impuls}% 
	\vec{v_1} - \vec{v_1'} = \frac{m_2}{m_1} (\vec{v_2'} - \vec{v_2})
\end{equation}

\lstset{language=TeX}% C, TeX, Bash, Python 
\begin{lstlisting}[
	%caption={}, label={code:}%% anpassen
]
% Mathe
\begin{equation}
	\label{eq:ekin}% 
	\sum E_{kin} = \sum E'_{kin}
\end{equation}
\end{lstlisting}


\clearpage
\section*{Ausblick}

Daraus ergeben sich gemäß (\autoref{tab:schritte}) \verb|(\autoref{tab:schritte})| folgende nächste Schritte, 
deren sequenzielle Ausführung von essenzieller Bedeutung ist.

\begin{table}[!ht]% hier: !ht
	\centering
	\begin{tabular}{@{}cl@{}}% lcr
		\toprule
		\textbf{Nr.} & \textbf{Vorgehen} \\
		\midrule
		1 & Aktuellen Forschungsstand recherchieren \\
		2 & Methoden entwickeln \\
		3 & Schlussfolgerung aufstellen \\
		\bottomrule
	\end{tabular}
	\caption{Nächste Schritte}\label{tab:schritte}
\end{table}

\clearpage
\lstset{language=TeX}% C, TeX, Bash, Python 
\begin{lstlisting}[
	%caption={}, label={code:}%% anpassen
]
% Tabelle
\begin{table}[!ht]% hier: !ht
	\centering
	\begin{tabular}{@{}cl@{}}% lcr
		\toprule
		\textbf{Nr.} & \textbf{Vorgehen} \\
		\midrule
		1 & Aktuellen Forschungsstand recherchieren \\
		2 & Methoden entwickeln \\
		3 & Schlussfolgerung aufstellen \\
		\bottomrule
	\end{tabular}
	\caption{Nächste Schritte}\label{tab:schritte}
\end{table}
\end{lstlisting}
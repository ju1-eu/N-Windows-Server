%ju 20-Feb-22 01-Keywords-Windows-Server.tex
\section{Microsoft Windows Server}\label{microsoft-windows-server}

Quelle: Udemy Patrick Gruenauer \footnote{\url{https://www.udemy.com/course/microsoft-windows-server-fur-einsteiger/}}

MMOGA \footnote{\url{https://www.mmoga.de}} Windows 11 Professional OEM
Key:

Windows Download \footnote{\url{https://www.microsoft.com/de-de/software-download/windows11}}
bootfähigen USB-Sticks erstellen \footnote{\url{https://rufus.ie/de/}}

\subsection{Voraussetzung Software und
Hardware}\label{voraussetzung-software-und-hardware}

Suche: \verb|pc-info| \textbf{Windows Voraussetzung}:
Win 10/11 Pro

Powershell mit adminrechten öffnen: \verb|systeminfo|
(\textbf{Anforderung für Hyper-V muss erfüllt sein.})

Suche: \textbf{Windows Features aktivieren}:
\verb|Hyper-V|

\textbf{Download Server und Client}

\emph{Windows Server Evaluierungsversion}: Windows Server 2022, englisch
(Standard), \textbf{180 Tage}, Microsoft Konto

\textbf{6x} 180 Tage \textbf{verlängern} (3 Jahre) in Powershell:
\verb|slmgr -rearm|

\emph{Windows 10 Evaluierungsversion}: Windows 10 Enterprise, deutsch,
\textbf{90 Tage}, Microsoft Konto

\subsection{Hyper-V-Manager öffnen}\label{hyper-v-manager-oeffnen}

\begin{enumerate}
\item
  $\to$ Hyper-V-Manager: \textbf{LAP} / (re. Mausklick)

  \begin{itemize}
  \item
    $\to$ \textbf{Manager für virtuelle Switches}

    \begin{itemize}
    \item
      Neuer Virtueller Netzwerkswitch / Intern $\to$ Virtuellen Switch
      erstellen / \verb|kurs|
    \end{itemize}
  \end{itemize}
\item
  $\to$ Hyper-V-Manager: \textbf{LAP} / (re. Mausklick)

  \begin{itemize}
  \item
    $\to$ \textbf{Neu / Virtueller Computer erstellen}

    \begin{itemize}
    \item
      Name und Pfad angeben: \verb|Server01|
    \item
      Generation angeben: \verb|Generation 2|
    \item
      Speicher zuweisen: \verb|4096| RAM
    \item
      Netzwerk konfigurieren: Verbindung: \verb|kurs|
      (Virtueller Netzwerkswitch)
    \item
      Installationsoptionen: ISO:
      \verb|Windows Server 2022.iso| auswählen
    \end{itemize}
  \item
    $\to$ Virtueller Computer: \textbf{Server01} / (re. Mausklick)
    Einstellungen

    \begin{itemize}
    \item
      Prüfpunkte deaktivieren
    \item
      CPU: \verb|2 Kerne| (min.)
    \end{itemize}
  \end{itemize}
\item
  $\to$ Hyper-V-Manager: \textbf{Server01} (doppelklick)

  \begin{itemize}
  \item
    \textbf{Setup}: Windows Server 2022 installieren

    \begin{itemize}
    \item
      Version: Standard, Desktop
    \item
      Name: \verb|administrator| key:
    \end{itemize}
  \end{itemize}
\item
  $\to$ Hyper-V-Manager: \textbf{LAP} / (re. Mausklick)

  \begin{itemize}
  \item
    $\to$ \textbf{Neu / Virtueller Computer erstellen}

    \begin{itemize}
    \item
      Name und Pfad angeben: \verb|Server02|
    \item
      Generation angeben: \verb|Generation 2|
    \item
      Speicher zuweisen: \verb|4096| RAM
    \item
      Netzwerk konfigurieren: Verbindung: \verb|kurs|
      (Virtueller Netzwerkswitch)
    \item
      Installationsoptionen: ISO:
      \verb|Windows Server 2022.iso| auswählen
    \end{itemize}
  \item
    $\to$ Virtueller Computer: \textbf{Server01} / (re. Mausklick)
    Einstellungen

    \begin{itemize}
    \item
      Prüfpunkte deaktivieren
    \item
      CPU: \verb|2 Kerne| (min.)
    \end{itemize}
  \end{itemize}
\item
  $\to$ Hyper-V-Manager: \textbf{Server02} (doppelklick)

  \begin{itemize}
  \item
    \textbf{Setup}: Windows Server 2022 installieren

    \begin{itemize}
    \item
      Version: Standard, Desktop
    \item
      Name: \verb|administrator| key:
    \end{itemize}
  \end{itemize}
\item
  $\to$ Hyper-V-Manager: \textbf{LAP} / (re. Mausklick)

  \begin{itemize}
  \item
    $\to$ \textbf{Neu / Virtueller Computer erstellen}

    \begin{itemize}
    \item
      Name und Pfad angeben: \verb|Client01|
    \item
      Generation angeben: \verb|Generation 2|
    \item
      Speicher zuweisen: \verb|4096| RAM
    \item
      Netzwerk konfigurieren: Verbindung: \verb|kurs|
      (Virtueller Netzwerkswitch)
    \item
      Installationsoptionen: ISO:
      \verb|Windows 10 Enterprise.iso| auswählen
    \end{itemize}
  \item
    $\to$ Virtueller Computer: \textbf{Client01} / (re. Mausklick)
    Einstellungen

    \begin{itemize}
    \item
      Prüfpunkte deaktivieren
    \item
      CPU: \verb|2 Kerne| (min.)
    \end{itemize}
  \end{itemize}
\item
  $\to$ Hyper-V-Manager: \textbf{Client01} (doppelklick)

  \begin{itemize}
  \item
    \textbf{Setup}: Windows 10 Enterprise installieren

    \begin{itemize}
    \item
      Netzwerk: kein Internet Auswahl
    \item
      kein Microsoft Konto verbinden: Weiter mit eingeschränktem Setup
    \item
      Name: \verb|admin| key:
    \item
      Windows anmelden
    \end{itemize}
  \item
    \textbf{Problem}: Windows-Lizenz ist abgelaufen
  \item
    $\to$ Hyper-V-Manager: Client01 / (re. Mausklick) Einstellungen

    \begin{itemize}
    \item
      Netzwerkkarte: virtueller Switch:
      \verb|Default-Switch| Auswahl
    \end{itemize}
  \item
    Hyper-V-Manager: \textbf{Client01} (doppelklick)
  \item
    \textbf{Windows aktivieren}: Netzwerk / Internet / Neustart
  \item
    Kontrolle: Windows Lizenz \textbf{90 Tage gültig}
  \item
    $\to$ Hyper-V-Manager: Client01 / (re. Mausklick) Einstellungen

    \begin{itemize}
    \item
      Netzwerkkarte: virtueller Switch: \verb|kurs|
      wieder umstellen
    \end{itemize}
  \end{itemize}
\end{enumerate}

\subsection{Grundkonfiguration Server01 - Server02 -
Client01}\label{grundkonfiguration-server01-server02-client01}

\begin{enumerate}
\item
  Hyper-V-Manager: \textbf{Server01} (doppelklick) \textbf{Anmeldung}
  user: \verb|administrator| key:

  \begin{itemize}
  \item
    Server-Manager / Local Server
  \item
    $\to$ \textbf{Computer name}: \verb|Server01|
  \item
    Neustart
  \item
    Server Manager / Local Server
  \item
    \textbf{Ethernet}: $\to$ Netzwerkkarte (rechte Mausklick) /
    Eigenschaften / Internetprotokoll

    \begin{itemize}
    \item
      IP address: \verb|192.168.0.1| Subnet mask:
      \verb|255.255.255.0|
    \end{itemize}
  \item
    Test in Powershell: \verb|ipconfig|
  \end{itemize}
\item
  Hyper-V-Manager: \textbf{Server02} (doppelklick) \textbf{Anmeldung}
  user: \verb|administrator| key:

  \begin{itemize}
  \item
    Server-Manager / Local Server
  \item
    $\to$ \textbf{Computer name}: \verb|Server02|
  \item
    Neustart
  \item
    Server Manager / Local Server
  \item
    \textbf{Ethernet}: $\to$ Netzwerkkarte (rechte Mausklick) /
    Eigenschaften / Internetprotokoll

    \begin{itemize}
    \item
      IP address: \verb|192.168.0.2| Subnet mask:
      \verb|255.255.255.0|
    \end{itemize}
  \item
    Test in Powershell: \verb|ipconfig|
  \end{itemize}
\item
  Hyper-V-Manager: \textbf{Client01} (doppelklick) \textbf{Anmeldung}
  user: \verb|admin| key:

  \begin{itemize}
  \item
    Suche: \verb|pc-infos| $\to$ Diesen PC
    umbenennen (fortgeschritten): Ändern

    \begin{itemize}
    \item
      $\to$ \textbf{Computer name}: \verb|Client01|
    \end{itemize}
  \item
    Neustart
  \item
    \textbf{Netzwerk und Interneteinstellungen} / Adapteroptionen ändern
    /
  \item
    \textbf{Ethernet}: $\to$ Netzwerkkarte (rechte Mausklick) /
    Eigenschaften / Internetprotokoll

    \begin{itemize}
    \item
      IP address: \verb|192.168.0.100| Subnet mask:
      \verb|255.255.255.0|
    \end{itemize}
  \item
    Test in Powershell: \verb|ipconfig|
  \end{itemize}
\end{enumerate}

$\to$ Hyper-V-Manager: Je \textbf{Server01}, \textbf{Server02} und
\textbf{Client01} (doppelklick) \textbf{Anmeldung} user:
\verb|administrator| bzw.
\verb|admin| (Client01) key:

\begin{itemize}
\item
  Suche: \textbf{Firewall} $\to$ Windows Defender Firewall mit
  erweiterter Sicherheit

  \begin{itemize}
  \item
    Eingehende Regeln:

    \begin{itemize}
    \item
      Datei- und Druckerfreigabe Echoanforderung ip4 und ip6 (\textbf{2x
      Regeln aktivieren})
    \end{itemize}
  \end{itemize}
\end{itemize}

Suche: \textbf{Powershell}

\lstset{language=Python}% C, TeX, Bash, Python 
\begin{lstlisting}[
	%caption={}, label={code:}%% anpassen
][language=bash]
# Test in Powershell
ping 192.168.0.1
ping 192.168.0.2
ping 192.168.0.100
ipconfig /all
hostname 
winver
netstat -a -p TCP
\end{lstlisting}

\subsection{Windows Server}\label{windows-server}

$\to$ Hyper-V-Manager: \textbf{Server01} (doppelklick)
\textbf{Anmeldung} user: \verb|administrator| key:

\begin{enumerate}
\item
  \textbf{Server-Manager}/

  \begin{itemize}
  \item
    Local Server/ $\to$ z. B. Zeitzone: Berlin
  \item
    Dashboard
  \item
    Verwalten
  \item
    Tools
  \end{itemize}
\item
  \textbf{Windows Admin Center}

  \begin{itemize}
  \item
    Download Windows Admin Center \footnote{\url{https://www.microsoft.com/de-de/windows-server/windows-admin-center}}
    vom $\to$ Host (Internet)
  \item
    kopieren nach $\to$ Server01 und installieren
  \item
    \textbf{Browser} öffnen und \url{https://server01} eingeben
  \end{itemize}
\end{enumerate}

\subsection{Rollen und Features
installieren}\label{rollen-und-features-installieren}

$\to$ Hyper-V-Manager: \textbf{Server02} (doppelklick)
\textbf{Anmeldung} user: \verb|administrator| key:

\begin{enumerate}
\item
  Server-Manager / Dashboard

  \begin{enumerate}
  \def\labelenumii{\arabic{enumii}.}
  \item
    \textbf{Rollen und Features hinzufügen}
  \item
    3 x Weiter bis Serverrollen $\to$ \textbf{Web-Server} wählen
  \item
    Weiter klicken bis zur Bestätigung $\to$ Installieren
  \item
    \textbf{Browser} öffnen und \url{http://localhost} eingeben
  \end{enumerate}
\end{enumerate}

\subsection{Active Directory
Domänendienste}\label{active-directory-domaenendienste}

$\to$ Hyper-V-Manager: \textbf{Server01} (doppelklick)
\textbf{Anmeldung} user: \verb|administrator| key:

\begin{enumerate}
\item
  Server-Manager / Lokaler Server / Verwalten

  \begin{enumerate}
  \def\labelenumii{\arabic{enumii}.}
  \item
    \textbf{Rollen und Features hinzufügen}
  \item
    3 x Weiter bis Serverrollen $\to$ \textbf{Active Directory Doman
    Services} wählen
  \item
    Weiter klicken bis zur Bestätigung $\to$ Installieren
  \end{enumerate}
\item
  Server-Manager / Dashboard / Benachrichtigungen

  \begin{enumerate}
  \def\labelenumii{\arabic{enumii}.}
  \item
    \textbf{Server zu einem Domain-Controller hochstufen}
  \item
    Neue Gesamtstruktur hinzufügen: Domänenname
    \verb|pagr.inet|
  \item
    key:
  \item
    4 x Weiter klicken bis zur Bestätigung $\to$ Installieren
  \item
    Neustart
  \item
    Anmeldung (\textbf{Server01})
  \end{enumerate}
\end{enumerate}

$\to$ Hyper-V-Manager: \textbf{Server02} (doppelklick)
\textbf{Anmeldung} user: \verb|administrator| key:

\textbf{Server Manager / Local Server}

\begin{enumerate}
\item
  \textbf{Ethernet}: $\to$ Netzwerkkarte (rechte Mausklick) /
  Eigenschaften / Internetprotokoll

  \begin{itemize}
  \item
    DNS-Server: \verb|192.168.0.1|
    (\textbf{Server01})\
  \end{itemize}
\item
  \textbf{Computername}: $\to$ Domäne:
  \verb|pagr.inet|

  \begin{itemize}
  \item
    Domäne beitreten: user und key von (\textbf{Server01})
  \item
    Neustart
  \item
    \textbf{Anmeldung} user:
    \verb|pagr\administrator| key: von
    (\textbf{Server01})
  \item
    Test in Powershell:
    \verb|Test-ComputerSecureChannel -Verbose| (True)
  \end{itemize}
\end{enumerate}

$\to$ Hyper-V-Manager: \textbf{Client01} (doppelklick)
\textbf{Anmeldung} user: \verb|admin| key:

\begin{enumerate}
\item
  \textbf{Netzwerk und Interneteinstellungen} / Adapteroptionen ändern /

  \begin{itemize}
  \item
    \textbf{Ethernet}: $\to$ Netzwerkkarte (rechte Mausklick) /
    Eigenschaften / Internetprotokoll /

    \begin{itemize}
    \item
      DNS-Server: \verb|192.168.0.1|
      (\textbf{Server01})
    \end{itemize}
  \item
    Test in Powershell: \verb|ping pagr.inet|
  \end{itemize}
\item
  \textbf{Domänen beitritt}

  \begin{itemize}
  \item
    Suche: \verb|pc-infos| $\to$ Diesen PC
    umbenennen (fortgeschritten): \emph{Ändern}

    \begin{itemize}
    \item
      $\to$ Domäne: \verb|pagr.inet|\
    \item
      Domäne beitreten: user und key von (\textbf{Server01})
    \end{itemize}
  \item
    Neustart
  \item
    \textbf{Anmeldung} user:
    \verb|pagr\administrator| key: von
    (\textbf{Server01})
  \end{itemize}
\end{enumerate}

\subsection{Active Directory - Benutzer Computer Gruppen und
Richtlinien}\label{active-directory-benutzer-computer-gruppen-und-richtlinien}

$\to$ Hyper-V-Manager: \textbf{Server01} (doppelklick)
\textbf{Anmeldung} user: \verb|administrator| key:

Server-Manager / Dashboard / \textbf{Tools} / $\to$ Active Directory -
Benutzer und Computer / Ansicht / Erweiterte Features

\begin{enumerate}
\item
  \textbf{Benutzer erstellen}

  \begin{itemize}
  \item
    \verb|pagr.inet| (re. Mausklick) Neu /
    Organisationseinheit: \textbf{HR} erstellen
  \item
    \textbf{HR}: (re. Mausklick) Neu / User

    \begin{itemize}
    \item
      Name: \verb|Franz Bizeps|
    \item
      Benutzeranmeldenname: f.bizeps@pagr.inet
    \item
      Passwort muss nach der ersten Anmeldung geändert werden!
    \end{itemize}
  \item
    $\to$ Hyper-V-Manager: \textbf{Client01} (doppelklick)
  \item
    \textbf{Anmeldung} user:
    \verb|pagr\administrator| key: von
    (\textbf{Server01})
  \item
    Suche: \verb|lusrmgr.msc| (Lokale Benutzer und
    Gruppen)

    \begin{itemize}
    \item
      Gruppen / Remotedesktopbenutzer: Objektnamen:
      \verb|Franz Bizeps| (Namen überprüfen)
    \end{itemize}
  \item
    Abmelden
  \item
    \textbf{Anmeldung} user: \verb|f.bizeps| key:
  \item
    Test in Powershell:

    \begin{itemize}
    \item
      \verb|Test-ComputerSecureChannel| (True)
    \item
      \verb|$env:logonserver|
      (\textbackslash server01)
    \end{itemize}
  \end{itemize}
\item
  \textbf{Benutzer konfigurieren}

  \begin{itemize}
  \item
    $\to$ Hyper-V-Manager: \textbf{Server01} (doppelklick)
  \item
    \textbf{Anmeldung} user: \verb|administrator|
    key:
  \item
    Server-Manager / Dashboard / \textbf{Tools} /

    \begin{itemize}
    \item
      $\to$ Active Directory - Benutzer und Computer / Ansicht /
      Erweiterte Features
    \end{itemize}
  \item
    \verb|pagr.inet|
  \item
    \textbf{HR} / \textbf{Franz Bizeps} (re. Mausklick) Eigenschaften
  \end{itemize}
\item
  \textbf{Computerkonten}

  \begin{itemize}
  \item
    \verb|pagr.inet| (re. Mausklick) Neu /
    Organisationseinheit: \textbf{Windows Server} erstellen
  \item
    \verb|pagr.inet| (re. Mausklick) Neu /
    Organisationseinheit: \textbf{Windows Clients} erstellen
  \item
    \textbf{Computers} / Client01 $\to$ \textbf{Windows Clients}
    verschieben
  \item
    \textbf{Computers} / Server02 $\to$ \textbf{Windows Server}
    verschieben
  \end{itemize}
\item
  \textbf{Gruppen erstellen und konfigurieren}

  \begin{itemize}
  \item
    \verb|pagr.inet| (re. Mausklick) Neu /
    Organisationseinheit: \textbf{Gruppen} erstellen
  \item
    \textbf{Gruppen} (re. Mausklick) Neu / Gruppe

    \begin{itemize}
    \item
      Gruppenname: HR
    \end{itemize}
  \item
    \textbf{HR} / \textbf{Franz Bizeps} (re. Mausklick) Eigenschaften

    \begin{itemize}
    \item
      Mitglied von: Hinzufügen / Objektnamen:
      \verb|HR| (Namen überprüfen)
    \end{itemize}
  \end{itemize}
\item
  einfache Gruppenrichtlinien erstellen
\end{enumerate}

\subsection{Netzwerkdienste mit Windows Server
bereitstellen}\label{netzwerkdienste-mit-windows-server-bereitstellen}

$\to$ Hyper-V-Manager: \textbf{Server01} (doppelklick)
\textbf{Anmeldung} user: \verb|administrator| key:

Server-Manager / Dashboard / \textbf{Tools} / $\to$ DNS

\begin{enumerate}
\item
  \textbf{DNS in Active Directory}

  \begin{itemize}
  \item
    Server01 / Forward-Lookupzonen
  \item
    \verb|pagr.inet| (re. Mausklick) $\to$ Neuer
    Host

    \begin{itemize}
    \item
      \verb|printer| und
      \verb|192.168.0.200|
    \end{itemize}
  \item
    \verb|pagr.inet| (re. Mausklick) $\to$ Neuer
    Alias

    \begin{itemize}
    \item
      \verb|drucker| und
      \verb|printer.pagr.inet|
    \end{itemize}
  \item
    Test in Powershell: \verb|nslookup|

    \begin{itemize}
    \item
      \verb|> drucker| (sollte auf printer.pagr.inet
      zeigen, Address: 192.168.0.200)
    \item
      \verb|> exit|
    \end{itemize}
  \end{itemize}
\item
  DNS auf Server02 manuell einrichten und Zonentransfer durchführen
  (Ergänzen)
\item
  \textbf{DHCP-Server Rolle auf Server02 installieren und konfigurieren}

  \begin{itemize}
  \item
    $\to$ Hyper-V-Manager: \textbf{Server02} (doppelklick)
  \item
    \textbf{Anmeldung} user:
    \verb|pagr\administrator| key:
  \item
    Server-Manager / Dashboard

    \begin{enumerate}
    \def\labelenumii{\arabic{enumii}.}
    \item
      \textbf{Rollen und Features hinzufügen}
    \item
      3 x Weiter bis Serverrollen $\to$ \textbf{DHCP} wählen
    \item
      Weiter klicken bis zur Bestätigung $\to$ Installieren
    \end{enumerate}
  \item
    Neustart
  \item
    Server-Manager / Dashboard / \textbf{Benachrichtigungen}

    \begin{itemize}
    \item
      $\to$ DHCP-Server Konfiguration abschließen
    \end{itemize}
  \item
    Server-Manager / Dashboard / \textbf{Tools} / $\to$ DHCP
  \item
    IPv4 (re. Mausklick) $\to$ Neuer Bereich erstellen

    \begin{itemize}
    \item
      Name: \verb|Clients|
    \item
      Start IP-address: 192.168.0.100 End IP-address: 192.168.0.200
    \item
      weiter bis Abschluß
    \end{itemize}
  \item
    Hyper-V-Manager: \textbf{Client01} (doppelklick)
  \item
    \textbf{Anmeldung} user:
    \verb|pagr\administrator| key:
  \item
    \textbf{Netzwerk und Interneteinstellungen} / Adapteroptionen ändern
    /

    \begin{itemize}
    \item
      \textbf{Ethernet}: $\to$ Netzwerkkarte (rechte Mausklick) /
      Eigenschaften / Internetprotokoll
    \item
      IP-Adresse automatisch beziehen
    \item
      DNS-Serveradresse automatisch beziehen
    \end{itemize}
  \item
    Test in Powershell: \verb|ipconfig -all| (Range:
    192.168.0.100 - 192.168.0.200)
  \end{itemize}
\item
  \textbf{Server02 als File-Server einsetzen}

  \begin{itemize}
  \item
    $\to$ Hyper-V-Manager: \textbf{Server02} (doppelklick)
  \item
    \textbf{Anmeldung} user:
    \verb|pagr\administrator| key:
  \item
    Server-Manager / Dashboard

    \begin{enumerate}
    \def\labelenumii{\arabic{enumii}.}
    \item
      \textbf{Rollen und Features hinzufügen}
    \item
      3 x Weiter bis Serverrollen
    \item
      $\to$ Datei- und Speicherdienste / Dateiserver und Dateiserver
      Resource Manager für Dateiserver wählen
    \item
      Weiter klicken bis zur Bestätigung $\to$ Installieren
    \end{enumerate}
  \item
    \textbf{Ordner erstellen} \verb|c:\Daten|
  \item
    \textbf{Ordner freigeben} \verb|Daten| (re.
    Mausklick) / Eigenschaften / Freigabe / Erweitere Freigabe

    \begin{itemize}
    \item
      Diesen Ordner freigeben
    \item
      Berechtigungen / Vollzugriff
    \end{itemize}
  \item
    Test: Suche: \verb|\\server02\Daten|
  \item
    Suche: Gruppenrichtlinienverwaltung (Ergänzen)
  \item
    $\to$ Hyper-V-Manager: \textbf{Client01} (doppelklick)
  \item
    \textbf{Anmeldung} user: \verb|f.bizeps| key:
  \item
    Windows Explorer öffnen / Netzwerklaufwerk prüfen
    \verb|x:\\server02\daten|
  \end{itemize}
\item
  Server02 als Druck-Server einrichten (Ergänzen)
\end{enumerate}

\subsection{Windows Server Sicherheit}\label{windows-server-sicherheit}

\begin{enumerate}
\item
  \textbf{Windows Server Backup auf Server02 installieren}

  \begin{itemize}
  \item
    $\to$ Hyper-V-Manager: \textbf{Server02} (doppelklick)
  \item
    \textbf{Anmeldung} user:
    \verb|pagr\administrator| key:
  \item
    Server-Manager / Dashboard

    \begin{enumerate}
    \def\labelenumii{\arabic{enumii}.}
    \item
      \textbf{Rollen und Features hinzufügen}
    \item
      4 x Weiter bis Features $\to$ \textbf{Windows Server-Sicherung}
      wählen
    \item
      Weiter klicken bis zur Bestätigung $\to$ Installieren
    \end{enumerate}
  \item
    $\to$ Hyper-V-Manager: \textbf{Server01} (doppelklick)
  \item
    \textbf{Anmeldung} user: \verb|administrator|
    key:
  \item
    \textbf{Ordner erstellen} \verb|c:\Backup|
  \item
    \textbf{Ordner freigeben} \verb|Backup| (re.
    Mausklick) / Eigenschaften / Freigabe / Erweitere Freigabe

    \begin{itemize}
    \item
      Diesen Ordner freigeben
    \item
      Berechtigungen / Vollzugriff
    \end{itemize}
  \item
    Test: Suche: \verb|\\server02\Daten|
  \end{itemize}
\item
  \textbf{Einen Ordner von Server02 auf Server01 sichern}

  \begin{itemize}
  \item
    $\to$ Hyper-V-Manager: \textbf{Server02} (doppelklick)
  \item
    \textbf{Anmeldung} user:
    \verb|pagr\administrator| key:
  \item
    Server-Manager / Dashboard / Tools
  \item
    \textbf{Windows Server-Sicherung} / $\to$ Assistent
    Sicherungszeitplan

    \begin{itemize}
    \item
      Benutzerdefiniert
    \item
      Elemente auswählen: \verb|Daten| (Ordner)
    \item
      Sicherungszeit: \verb|21:00|,
      \verb|Einmal pro Tag|
    \item
      Sicherung auf einem freigegeben Netzwerkordner erstellen
    \item
      Pfad: \verb|\\server01\Backup|
    \item
      \textbf{Berechtigung für das planen der Sicherung} user:
      \verb|pagr\administrator| key:
    \end{itemize}
  \item
    Suche: Aufgabenplanung (Ergänzen)
  \end{itemize}
\item
  \textbf{Windows Updates manuell ausführen}

  \begin{itemize}
  \item
    $\to$ Hyper-V-Manager: \textbf{LAP} / (re. Mausklick)

    \begin{itemize}
    \item
      $\to$ \textbf{Manager für virtuelle Switches}
    \item
      Virtuellen Switch \verb|kurs| / \textbf{Intern}
      umstellen $\to$ \textbf{Externes Netzwerk} (Internet)
    \end{itemize}
  \end{itemize}
\end{enumerate}
